% -*- coding: utf-8; ispell-dictionary: "french"; -*-

%------------
% Conclusion
%------------

L'utilisation de méthodes formelles pour la validation de programmes n'est pas
une activité récente. Cependant, les industriels ayant toujours eu recours aux
tests pour valider un programme, il existe une inertie dans ce domaine
qui bloque la propagation d'autres approches.
Il y a néanmoins un regain d'intérêt pour les méthodes formelles
depuis quelques années, la fiabilité des composants formellement
validés étant supérieure à celle des composants testés. C'est surtout
le cas dans les domaines critiques tel que la santé, les systèmes de
transport, ou la production d'énergie, qui recquièrent un niveau
d'exigence élevé pour la validation des programmes utilisés.\\

Scade est un outil très utilisé pour le développement de composants
pour les logiciels embarqués. Mais la validation de ces composants
passe encore par des tests, car il n'y a aucune validation par méthode
formelle integrée à l'outil de développement. L'intérêt de ce projet
est de proposer une validation par méthode formelle en utilisant B, ce
qui est rendu possible par une traduction automatique des composants Scade
vers des machines B. 
Ainsi, dans l'état actuel, le traducteur permet un gain de temps et de
sécurité dans la validation de composants développés avec Scade.\\

Il reste cependant à formaliser une preuve de correction de la
traduction elle même. On pourra réfléchir à une démonstration basée
sur la structure des différentes équations traduites.\\
De plus, ce travail ne s'appuie que sur un fragment de Scade, les possibilités
offertes par le langage étant très vastes. Ainsi, à partir de ce
projet, il est envisageable de poursuivre différents axes pour élargir
le fragment de Scade traduit. On pourrait notamment s'intéresser aux
machines à état, qui permettent de décrire des automates. On peut
également réfléchir à la possibilité de traduire d'autres opérateurs
synchrones en B. 

