% -*- coding: utf-8; ispell-dictionary: "french"; -*-

%--------------
% Introduction
%--------------


Ce stage s'est déroulé au sein d'un projet financé par l'Agence
Nationale de la Recherche: CERCLES2. Ce projet a pour but la
certification compositionnelle des logiciels embarqués critiques et
sûrs. C'est la notion de composant réutilisable et assemblable pour former des
logiciels critiques et sûrs qui est à la base du projet, l'intérêt étant à la fois
pratique par le gain de temps et d'effort, et économique. \\

Un acteur majeur du développement de systèmes embarquées critiques est Scade,
un acronyme pour Safety Critical Application Developpement Environment. Cet
environnement de développement est basé sur la programmation graphique, par
schémas-blocs, permettant de définir des programmes faciles à lires et
permettant d'engendrer directement du code compilable (C ou ADA). Il est
notamment utilisé en aéronautique (grande partie du logiciel embarqué de
l'A380), dans le domaine spatial ou dans le nucléaire. C'est donc avec Scade
que sont écrits les composants, les contrats étants rédigés sous forme
textuelle en accompagnement du composant.\\

INSERER CAPTURE SCADE\\

Pour assurer que ces composants sont sûrs et réutilisables, on
utilise une méthode formelle, qui permet d'exprimer la signification d'un
composant dans un formalisme mathématique, afin de démontrer leur
validité par rapport à une spécification.\\
Il faut alors introduire le concept des contrats: un contrat est
associé à un composant et impose des conditions sur ses entrées
(pré-conditions) et sur ses sorties (post-conditions). Ils donneront
ainsi une spécification du composant. Les contrats
ont été introduits par C.A.R Hoare, qui donne la définition
suivante:\\

\noindent
\fbox{
\begin{minipage}{\textwidth}
\begin{center}
$P\{Q\}R$ \\
\emph{"If the assertion P is true before initiation of a program Q, then the
assertion R will be true on its completion"}
\end{center}
\end{minipage}
}\\

A partir d'un composant et de son contrat, il faut alors vérifier formellement
que:
\begin{itemize}
\item  (i) la définition du composant satisfait le contrat
\item (ii) l'utilisation du composant satisfait les pré-condition, et en
conséquence de (i) le résultat satisfait les post-conditions.
\end{itemize}
La validation est alors faite par une démonstration formelle.\\

Il existe d'autres méthodes formelles, basées sur des règles de typage
des programmes, introduites par la correspondance de
Curry-Howard dans à la fin des années 50.
L'avantage de la méthode choisie, la méthode B, est qu'elle a déjà
fait ses preuves industriellement, elle a notamment été utilisée pour
développer la ligne METEOR (ligne 14) du métro parisien, qui est
entièrement automatisée.\\ 
Elle a été introduite
par J.R. Abrial dans les années 80. Elle est basé sur le raffinement de
spécifications formelles vers une spécification exécutable. La spécification
formelle est rédigée dans un formalisme mathématique de haut niveau appelé
machine abstraite, dont le principe de calcul est basé sur le calcul
des prédicats du premier ordre étendu avec une théorie des
ensembles. Le raffinement de cette machine abstraite consiste à la
reformuler de façon plus concrète et à l'enrichir avec des
substitutions correspondants aux instructions du composant. Ce
raffinement de plus bas niveau est appelé implantation. Il peut y
avoir des raffinements intermédiaires, mais nous n'aurons besoin que
du raffinement de la machine abstraite vers l'implantation. Chaque
étape de raffinement passe par une étape d'obligations de preuves, une
validation par démonstration formelle, garantissant la fidélité de la
spécification raffinée par rapport à la spécification originale. \\ 

Mon travail fut de développer un traducteur permettant de passer d'une méthode à
l'autre. Le traducteur suit une ligne de compilation classique, prenant en
entrée un code issu de Scade et produisant en sortie une machine abstraite
correspondant aux spécification du contrat, ainsi que la machine raffinée qui
implante le composant. \\

SCHEMA PRINCIPE GENERAL
