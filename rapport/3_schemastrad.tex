% -*- coding: utf-8; ispell-dictionary: "french"; -*-

%------------------------------------
% Chapter 3 - Schémas de traduction
%------------------------------------


Dans les deux parties précédentes, nous avons identifié les différents
éléments de Scade et de la méthode B dont nous avions besoin pour
établir la traduction. Cette partie défini les schémas de traduction
utilisés pour réaliser le traducteur.


...
Reprendre les schémas de traduction du rapport scade to B.

\section{Specification}
Signature des noeuds -> specification en B
Reprise des informations de typage
Reprise du contrat et traduction en PRE .. THEN .. END


\section{Implémentation}
Ajout des variables d'etats + initialisation (pour les registres)
Ajout des variables locales
tri topologique pour la séquence de substitution

\section{Le traducteur}
Développement du traducteur en Ocaml (pourquoi, avantages
inconvénients)
Comment a-t-il été développé
Parties sensibles...
