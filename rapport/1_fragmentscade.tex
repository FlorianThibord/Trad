% -*- coding: utf-8; ispell-dictionary: "french"; -*-

%----------------------------
% Chapter 1 - Fragment Scade
%----------------------------


Scade a été développé par le laboratoire Verimag à partir des travaux sur
le langage synchrone Lustre, puis repris par Esterel-technologie. On retrouve
ainsi les notions de Lustre dans le langage de Scade, un programme est découpé
en noeuds. Ces noeuds sont les composants que nous voulons traduire.
Les noeuds Scade considéré dans le cadre du projet Cercles2 sont
soumis à quelques restrictions. En effet, il faut limiter le langage
utilisé, car certains éléments du langage sont spécifiques aux
langages synchrones et ne sont donc pas traductibles en B.\\


% SECTION 1 : Temporalité

\section{Le temps avec Scade}

Le temps est un élément primordial dans ces systèmes dits réactifs, il est
découpé en instants discrets.\\ 
Le langage de Scade est basé sur la notion d'instant, 

\paragraph{Une horloge unique}
Ainsi, la première restriction à noter est qu'il n'y a qu'une seule
horloge, celle de base. Avec les langages synchrones, on peut
synchroniser des instructions sur des horloges différentes avec, par
exemple, l'opérateur when:

EXEMPLE/RESULTAT

Mais dans les programmes que nous manipulerons, il n'y aura donc pas
de définitions d'horloges, d'utilisation de when, ou current. Les
instructions d'un noeud sont toutes calculées sur l'horloge de base
uniquement.



\paragraph{Des registres}
La seconde restriction concerne l'utilisation des opérateurs pre et
->. Ce sont des opérateurs temporels:
\begin{itemize}
\item pre X donne la valeur de l'expression X à l'instant précédent. A
l'instant 0 \footnote{On suppose que le premier instant est l'instant
0}, la valeur de pre X n'est pas définie. 
\item A -> B donne au premier instant la valeur de l'expression A, 
puis la valeur de l'expression B pour les instants allant de 1 à n. 
\end{itemize}
On ne pourra utiliser que la construction suivante utilisant ces deux
opérateurs: 
\begin{center}
$A\rightarrow(pre~X)$
\end{center}
Cette construction correspond au bloc SIMULINK 1/Z, où A représente un
flux constant qui donnera la valeur de sortie à l'instant 0 du
bloc. Puis pour les instants 1 à n, on aura la valeur de l'expression
X à l'instant (1 à n)-1. \\
On appellera cette construction un registre, qui est initialisé avec la
valeur A, et qui permet d'accéder à la valeur précédente de X à tout
instant. Cette construction permet de donner un état à un composant.

LE COTE REGISTRE = ETAT A DEVELOPPER

+ Utilisation de l'opérateur FBY de scade? (pas dans lustre... comment est-il traduit?)
+ Si preuves de correction en aout, développer le problème langage synchrone
vers langage formel/impératif?


% SECTION 2 : Specification

\section{Contrat}

\paragraph{Assertions}
On peut définir des assertions dans un noeud afin de poser des
restrictions sur les valeurs d'entrée ou de sortie du composant. Avec
Scade, ces assertions sont possibles avec :
\begin{itemize}
\item \emph{assume x: expr}, où x est une des entrées du noeud, et
expr un prédicat portant sur cette entrée.
\item \emph{guarrantee x: expr}, où x est une des sorties du noeud, et
expr un prédicat portant sur cette sortie.
\end{itemize}
Ces assertions forment le contrat du composant, et seront
obligatoires sauf pour la restriction sur les booléen qui est triviale (la
valeur sera vraie ou fausse).\\

EXEMPLE


\paragraph{Concernant les types}
Ensuite, au niveau des types de données utilisées, on pourra manipuler
des entiers, réels et booléens. Et on pourra également manipuler des
tableaux de ces types. En revanche, les types définis par
l'utilisateur tels que les types enregistrement ne seront pas gérés
par le traducteur.\\

?HISTOIRES DE POLYMORPHISME POUR LES OPERATIONS SUR LES TABLEAUX A VOIR?


\section{Le langage d'entrée du traducteur}

Scade étant un environnement de programmation par schémas-blocs, on
développe avec des boîtes. Par exemple, une addition sur les fluxs A
et B s'écrit:

EXEMPLE D'UN PROGRAMME SIMPLE EN SCADE

Ce langage en boîtes est basé sur le langage synchrone Lustre, et
c'est le programme en Lustre que nous allons parser avec le
traducteur. La représentation graphique du programme est ainsi
réécrite en Lustre avant d'être traduite. C'est donc du code Lustre
que l'on traduit. 

REPRISE DE L'EXEMPLE EN LUSTRE

PUIS DETAILLER COMMENT OBTENIR LE LUSTRE?
(FICHIER SAOFD)

\paragraph{Expressions}
Les expressions disponibles sont toutes les expressions arithmétiques
(+, -, /, *, mod), les expressions relationnelles (<, >, <=, >=, =, <>)
et logiques (and, or, xor, not).
Les expressions conditionelles sont également possibles (if .. then
.. else ..),en revanche les constructions à base de l'opérateur case
ne sont pas acceptée. (A AJOUTER?) \\
Sont également disponibles les opérations sur les tableaux, telles que
la définition, l'index, et la concaténation.


\paragraph{Ajouts d'opérations au langage}

A REFORMULER...

Dans la situation où un utilisateur veux ajouter une opération qui
n'est pas reconnue par le traducteur, il peut définir cette opération
dans une bibliothèque Scade, et définir cette même opération dans une
machine B. Cette opération sera vue comme un appel de noeud.
Une fois la traduction terminée, il suffit d'indiquer en
haut du couple de machines B produites quelle machine à importer est
utilisée dans ce programme.

SCHEMA A INCLURE

