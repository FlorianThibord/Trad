% -*- coding: utf-8; ispell-dictionary: "french"; -*-
\documentclass{cercles2}
\usepackage{float}
\usepackage{alltt}
\usepackage{multicol}

% Macros locales
\newcommand{\cercles}{\textsf{CERCLES{\raisebox{0.5ex}{\small 2}}}}

% Définitions obligatoires (mais qui peuvent être "vides")

\Dtache{}
\Dtitre{Documentation du traducteur Scade2B}
\Dstatut{préliminaire} %préliminaire % en revue % final
\Ddate{octobre 2013}
\Dversion{0.1}
\Dorganisation{PPS}
\Dcourriel{florian.thibord@\ pps.univ-paris-diderot.fr}
\Dcontact{Florian Thibord}
\Dtelfax{Tél:~\mbox{+33 6 85 14 54 89}}

% Auteurs du document

\ajoutAuteur{\Ccontact}{\Corganisation}{\Ccourriel}
%% \ajoutAuteur{Florian Thibord}{PPS}{florian.thibord@\ pps.univ-paris-diderot.fr}

% Historique

\ajoutHistorique{0.1}{26~oct.~2013}{Création}{toutes}
%% \ajoutHistorique{0.2}{5~déc.~2012}{Paragraphe \textit{Schéma de
%%     traduction} et cosmétique}{toutes}
%% \ajoutHistorique{0.3}{5~déc.~2012}{Schéma de
%%     traduction 1 et 2; autres cosmétiques}{toutes}
%% \ajoutHistorique{\Cversion}{17~déc.~2012}{Contraintes annexes de
%%   typages et assertions; toujours du cosmétique}{toutes}
%% \ajoutHistorique{\Cversion}{13~sep.~2013}{Ajouts et synthèse en vue du rapport final}{toutes}

\begin{document}
\maketitle

\tableofcontents

\begin{quote}
  Cette documentation vise à décrire le fonctionnement et l'évolution
  du traducteur Scade2B.
\end{quote}


\chapter{Initialisation d'un registre à partir d'une entrée d'un composant}

\section{Exemple: \texttt{integrateur\_sature}}

Jusqu'à présent, les registres étaient initialisé uniquement avec des
constantes, or nous aimerions pouvoir utiliser une entrée du composant
pour initialiser un registre.

Considérons l'exemple \texttt{integrateur\_sature} suivant:

%% \begin{figure}[h]
%% \caption{nav1d/integrateur\_sature}
%% \label{fig:integ_sat_graph}
%% \vspace{-1ex}
%% \begin{center}
%% \includegraphics[scale=0.5]{integ_sat.pdf}
%% \end{center}
%% \vspace{-2ex}
%% \end{figure}

\emph{Planche à venir...}

Ce noeud contient une opération \texttt{fby}, dont l'initialisation
correspond à l'entrée \texttt{P\_INITIALISATION}. La version textuelle
de ce noeud est la suivante:

\begin{figure}[h]
\label{fig:integ_sat_text}
\caption{Version textuelle du composant \texttt{integrateur\_sature}}
\begin{verbatim}
node integrateur_sature(incr : int; P_INITIALISATION : int; P_SATURATION : int)
  returns (integr : int; integr_m1 : int)
var
  _L1 : int;
  _L2 : int;
  _L3 : int;
  _L7 : bool;
  _L6 : int;
  _L8 : int;
  _L9 : int;
let
  _L1= incr;
  integr= _L6;
  integr_m1= _L3;
  _L2= _L1 + _L3;
  _L3= fby(_L6; 1; _L8);
  _= _L7;
  _L6, _L7= #1 saturation(_L2, _L9);
  _L8= P_INITIALISATION;
  _L9= P_SATURATION;
tel
\end{verbatim}
\end{figure}

\newpage

Si on traduit ce noeud dans l'état actuel du traducteur, on obtiendra
le couple de machines B suivantes :

\begin{figure}[h]
\label{fig:integ_sat_B}
\caption{Machines B de \texttt{integrateur\_sature}}
\setlength{\columnseprule}{0.05cm}
\begin{multicols}{2}
\begin{verbatim}
MACHINE M_integrateur_sature
SEES M_Consts
OPERATIONS

integr,integr_m1<--integrateur_sature(
  incr,P_INITIALISATION,P_SATURATION)=
 PRE
   P_SATURATION : INT &
   P_INITIALISATION : INT &
   incr : INT
 THEN
   integr_m1 :: { ii | ii : INT }||
   integr :: { ii | ii : INT }
 END 
END
\end{verbatim}

\columnbreak

\begin{verbatim}
 IMPLEMENTATION M_integrateur_sature_i
 REFINES M_integrateur_sature
 SEES M_Consts
 IMPORTS M_saturation

 CONCRETE_VARIABLES L3
 INVARIANT 
    L3 : INT
 INITIALISATION 
    L3 := L8

 OPERATIONS

 integr,integr_m1<--integrateur_sature(
   incr,P_INITIALISATION,P_SATURATION)=
  VAR L1, L2, L7, L6, L8, L9 IN
    integr_m1 := L3; 
    L9 := P_SATURATION; 
    L8 := P_INITIALISATION; 
    L1 := incr; 
    L2 := L1 + L3; 
    L6, L7 <-- saturation(L2, L9); 
    integr := L6;
    L3 := L6
  END 
 END
\end{verbatim}
\end{multicols}
\end{figure}

L'implémentation n'est pas correctement initialisée. Dans la clause
\texttt{INITIALISATION}, la variable \texttt{L8} n'est pas définie,
car c'est une variable locale de l'opération qui reçoit le flux
d'entrée \texttt{P\_INITIALISATION}. 
\noindent
Plusieurs solutions permettent de résoudre ce problème:
\begin{itemize}
%% \item Solution 1 : Traduire le composant \texttt{integrateur\_sature} par une
%%   machine paramétrée, avec la variable \texttt{L8} comme paramètre, et
%%   supprimer cette dernière de la liste des variables locales.
\item Solution 1 : Déclarer une nouvelle variable d'état
  \texttt{temps}, de type \texttt{INT}, initialisée à 0, et réaliser
  l'initialisation du registre non plus dans la clause
  \texttt{INITIALISATION}, mais en début d'opération.
\item Solution 2 : A la place de la variable \texttt{L8} comme paramètre de la
  machine, utiliser l'entrée \texttt{P\_INITIALISATION}, et supprimer
  cette entrée de la liste des paramètres d'entrées de l'opération. 
\end{itemize} 

%% ANNULER SOLUTION 1, NE METTRE QUE LES SOLUTIONS 2 et 3!!!!!!!!!!!!!!!!!!!!!!!!!!!!!!!!!!!!!!!!!!!!!!!!!!!!!!!

\subsection{Solution 1}

En utilisant une substitution \textsf{Condition} en début d'opération,
on peut tester la valeur de la variable d'état \texttt{temps} et
affecter la valeur du registre en conséquence:\\
\begin{verbatim}
    IF temps = 0 THEN L3 := L8 ELSE L3 := L3 END;
\end{verbatim}
Or, en début d'opération, \texttt{L8} n'a pas encore été affectée, il
faut donc placer la substitution dans laquelle \texttt{L8} est
affectée d'une valeur avant la condition. Ainsi l'opération de cette
machine est correcte. \\

Cependant, quelle valeur donner au registre \texttt{L3} dans la clause
\texttt{INITIALISATION}? Cette valeur d'initialisation ne sera pas
prise en compte par l'opération car la première \texttt{L3} sera
immédiatement affectée par la valeur d'initialisation dans la
substitution \textsf{condition}. Mais pour les obligations de preuves,
il faut que les substitutions de la clause \texttt{INITIALISATION}
respectent l'\texttt{INVARIANT} de la machine. Si un invariant est lié au registre L3, 
on ne peut pas lui affecter une valeur d'initialisation
aléatoire.
Dans l'exemple, le registre est lié à la sortie
\texttt{integr\_m1}. Si on pose la condition suivante sur cette sortie:
\begin{verbatim}
    guarantee G_1 : integr_m1 < 10 and integr_m1 != 0 
\end{verbatim}
Elle sera traduite par l'invariant B:
\begin{verbatim}
    INVARIANT 
        L3 : INT & L3 < 10 & L3 <> 0 
\end{verbatim}
Il faut donc initialiser le registre à une valeur respectant cet
invariant, alors que cette valeur ne sera jamais utilisée.
Une solution n'ayant pas ce genre de contrainte est donc préférable.



\subsection{Solution 2}

En utilisant des machines paramétrées, on peut alors directement
initialiser les variables d'état de la  machine avec les paramètres
donnés. On a alors plus besoin d'une substitution \textsf{condition}
en début d'opération. 
Cette solution consiste à paramétrer la machine avec la variable
\texttt{L8}, qui ne sera plus une variable locale de l'opération mais
un paramètre de la machine. 
Pour que la machine IMPLEMENTATION soit correcte, il faut ajouter une
clause \texttt{CONSTRAINTS} dans laquelle on indique le type et les
restrictions liées aux différents paramètres de la machine, sous forme
de prédicat.

Dans la clause \texttt{INITIALISATION}, \texttt{L8} sera alors défini,
et les obligations de preuves s'appuieront sur le prédicat défini dans
la clause \texttt{CONSTRAINTS} pour vérifier la correction de
l'initialisation de la machine.

Dans l'opération, \texttt{L8} est affectée par l'entrée de l'opération
\texttt{P\_INITIALISATION}, et les deux


....
Réécrire avec P\_INITIALISATION en paramètre?


\section{Utilisation de machines paramétrées}



\end{document}
